\chapter{RELATED WORK}
\label{chap:related}
Based on the summary of $kSPR$\cite{tang_mouratidis_yiu_2017}, 
preference-based querying which based on the value of each attribute of products
are mainly two kinds, skyline \cite{borzsony_kossmann_stocker, papadias_tao_fu_seeger_2005, Efficient_Progeressive, kossmann_ramsak_rost_2002}
and top-$k$ query 
\cite{chang_bergman_castelli_li_lo_smith_2000, hristidis_papakonstantinou_2004, prefer, towards_robust, zou_chen_2008}. 
Skyline is also named as non-dominated set 
\cite{996017}, 
which including all 
the data that each of them isn't dominated by any other data in the dataset. "X dominates Y" 
means X is better them or equivalent to Y in all dimensions and there is at least one dimension 
that X is better than Y. In our paper, we are more closed to top-$k$ query.  Top-$k$ query 
will return $k$ products such that their scores are ranking top-$k$ respecting to the 
input user. Our problem is to find the region that where the new product lies 
will it rank top-$k$ for most of the rest given users. Another related problem is 
reverse top-$k$ \cite{vlachou_doulkeridis_xx, vlachou_doulkeridis_kotidis_norvag_2010, vlachou_doulkeridis_kotidis_norvag_2011}, 
which is to return the users that input product can rank top-$k$ 
respecting to them. Based on the output of top-$k$ query, why-not top-$k$ query 
\cite{6268270} is proposed
to change the user weight vector by advertisements, correcting wrong user information or other ways
with the minimum penalty to make an input product ranks top-$k$.  At the same time, 
based on the output of reverse top-$k$ query, why-not reverse top-$k$ query
\cite{gao_liu_chen_zheng_zhou_2015} is proposed to how to change the $k$ in top-$k$, 
the user weight vector $w$ or the product $p$'s attribute values so as to make $w$ shown in 
the reverse top-$k$ result of $p$.  


One of the related studies for our problem is $k-hit$ query\cite{peng_wong_2015}, which 
attempts to find 
$k$ products from given product dataset so as to rank top-1 of users as more as 
possible. The candidate solution of $k-hit query$ is discrete and finite while $kCRM$ 
is to 
find a new product ranks top-$k$ for as more users as possible.  At the same time, 
$CRM$ may return unknown number optimal products or even infinite products from continuous 
candidate space if only 
they are all optimal at the same time. 


A recent study $TopRR$\cite{tang_mouratidis_yiu_chen_2019} is very closed to our 
problem, which is attempting to return 
the product region that each of whose products can rank top-$k$ for all input users. 
The major different between $TopRR$ and $kCRM$ is that the candidate space is not complete 
in the domain of product for $kCRM$ since $kCRM$ can only choose the product that 
satisfies the constraint, 
which means in most cases, the optimal products can't cover all users and it is unknown
that the optimal products will cover how many or which users. 


In our paper, we use the $CTA$ approach as mentioned in $kSPR$ as our baseline. $kSPR$
is to return the user region that an input product can rank top-$k$ and it uses
a data structure $CellTree$ to exactly identify in which user space the product's score is
better than product dataset's another product's score, so $CellTree$ can take the regions that 
the input product not worse than other k products as results and return them. 
For $kCRM$, we transform our problem into return the product region that covers input users 
as more as possible. We find that to cover one or multiple users 
product also lies in regions and $CellTree$ can store and process them either. $CellTree$
will record each region cover how many users and return the region covers the most 
users as $kCRM$'s answer.   
 

 


