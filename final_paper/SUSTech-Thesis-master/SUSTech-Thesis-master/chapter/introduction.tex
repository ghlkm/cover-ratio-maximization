\chapter{INTRODUCTION}
\label{chp:introduction}
Take smartphone market as an example, there are different models of smartphones 
($D=$\{$r_1, r_2, ..., {r_m}$\}). Each model ($r_i=(r_i[1], r_i[2], ..., r_i[d])$) 
has different prices, pixels, battery capacities, cooling capacities and etc. 
Now a company $P$ owns $x$ models ($P=$\{$p_1, p_2, ..., p_x$\}$\in D$) 
and a user data set $W=$\{$w_1, w_2, ..., w_n$\}. Different users prefer in 
different aspects, for example some of them may prefer smartphones that with 
large battery capacity but some prefer those with better quality of screen, so each 
$w_i=(w_i[1], w_i[2], ..., w_i[d])$ represents the preference weight vector corresponding 
to a single user. The score of a model $r_i$ respecting to a user $w_i$ is the 
dot product $r_i\cdot w_i$. Usually a user will only make a choice from his own 
view of top-$k$, so a product ranks top-$k$ for the users is quite important. 
A product covers a user only when its score ranks top-$k$ among all existing 
products $D$. With the development of company and market, company has to develop a new 
product to cover more users and make more profit. But with the limitation of technology, 
money and other factors, one can't develop a perfect product to cover all users. 
It can only develop a product that covers as more user as possible under a constraint. 
For a company, some of its products have covered some users, so what it wants to do 
is how to make this new product cover more users that uncovered before.

In real life, k-Cover Ratio Maximization ($kCRM$) can solve the problems that how to 
decide the next generation product for companies. In advertising industry, it can help 
merchants how to cover specific group such as students, pregnant women and children. 
Besides it can tell advertiser where to set up new advertising board and if do so it 
can cover which group of people. Data analysts can use it to discover which group of 
people is ignored by the market. It can tell vedio makers to make which kinds of vedios
to attract users in YouTube, TikTop or other vedio platform.

Generally speaking, it is not a product covers a user but a group of products covers 
a user. And for different users, there are different groups of products. Among these 
groups of products, there are uncertain number of identity products. In our problem, 
we are aim to find this new product in continuous product space, which means there are 
infinite candidate products, so it is difficult to tell which product covers the most 
users.

In this paper, we will use computational geometric nature of $kCRM$ to explain how to 
find the exact optimal options (products) with the data structure $CellTree$ mentioned 
in $kSPR$\cite{tang_mouratidis_yiu_2017}. Besides, from some observations of this problem, we propose advance 
method that ignore irrelevant users and candidate products to save time. At last 
we sample products that satisfy the constraint and use their maximal cover count to 
prune the $CellTree$.

